\chapter{Sezione teorica}
<<descrizione del contesto teorico della tesi>>
\section{Simulazione delle masse}
 
\section {NetLogo} 
NetLogo è un linguaggio di programmazione e un ambiente di modellazione di sistemi complessi. Adatto per la simulazione e lo studio di fenomeni naturali e sociali che si evolvono nel tempo. Gli utenti possono dare istruzioni a migliaia di agenti in modo individuale o collettivo, permettendo quindi lo studio dei loro comportamenti su più livelli, come quello microscopico dei behaviors individuali o quello macroscopico delle loro interazioni con gli altri.\\
Nasce a scopo educativo e di ricerca, dalla fusione di \textbf{Logo} e \textbf{StarLisp}. Dal primo eredita il principio \textit{"low threshold , no ceiling"}, ovvero bassa soglia di conoscenza per il suo utilizzo, rendendolo accessibile a utenti inesperti nella programmazione, ma allo stesso tempo completa programmabilità rendendolo quindi anche uno strumento utile per la ricerca. Da Logo viene ereditato anche il concetto fondamentale di \textit{turtle}, con la differenza che Logo permetteva il controllo di un unico agente, mentre un modello NetLogo può averne migliaia. Da StarLisp, invece, NetLogo eredita i molteplici agenti e la loro \textit{concurrency}.\\
\subsection{Linguaggio}
Come Linguaggio NetLogo si evolve da Logo al quale aggiunge il concetto di agenti e di concurrency. In generale Logo è molto conoscuto per il concetto di \textbf{turtle} che ha introdotto. NetLogo generalizza questo permettendo il controllo di centinaia o migliaia di turtles che si muovono e interagiscono tra di loro.\\
Il mondo in cui i turtles si muovono è suddiviso in \textbf{patches} anche esse interamente programmabili, sia turtles che patches vengono chiamate collettivamente \textbf{agents}. Tutti gli agenti possono interagire tra di loro e eseguire istruzioni in modo concorrenziale. NetLogo include inoltre un terzo tipo di agente, l'\textbf{observer} il quale è unico. In generale l'observer è quello che impartisce ordini agli agenti.\\
Possono essere definite diverse “razze” (\textbf{breeds}) di turtles, ciascuna con variabili e behaviors caratteristici.\\
Una peculiarità che contraddistingue NetLogo dai suoi predecessori sono gli “agentsets”, ovvero insiemi di agenti.




 
