\chapter{Introduzione}
NetLogo è un linguaggio di programmazione agent-based e un ambiente di modellazione di sistemi complessi. Permette la simulazione di fenomeni naturali e sociali, e si adatta molto bene alla simulazione di eventi complessi che si evolvono nel tempo. La natura del linguaggio di programmazione permette di specificare comandi a singoli agenti o a intere \textit{breeds}, definendo quindi \textit{behaviors} individuali e globali. In questo modo si possono studiare comportamenti a più livelli, come quello microscopico dei singoli individui, o quello macroscopico delle loro interazioni e dei pattern che ne possono emergere.\\
NetLogo come linguaggio di programmazione multi-agent eredita da \textbf{Logo} il principio fondamentale " low threshold, no ceiling ". Low threshold implica la sua accessibilità a utenti poco esperti o con alcuna esperienza di programmazione precedente. No ceiling, invece, significa un apertura completa a progetti di ogni tipo, rendendolo quindi uno strumento molto utile per la ricerca.\\
La sua natura, però lo rende un linguaggio piuttosto rigido e, per alcuni aspetti, poco versatile ad occhi esperti, lo scopo di questo progetto, quindi è di agevolare l'utente NetLogo nella costruzione di particolari modelli. Si vuole facilitare la scrittura del codice NetLogo che permetta di rappresentare ambienti fisici sotto forma di grafi.\\
Il progetto prevede la lettura di un file XML in cui si descrive il modello di interesse, costituito da:
\begin{itemize}
\item ambiente fisico
\item comportamento degli attori
\item stato iniziale 
\end{itemize} 