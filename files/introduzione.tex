\chapter{Introduzione}
La simulazione delle masse negli ultimi decenni ha attirato l'attenzione di un numero crescente di gruppi di ricerca. Gli ambiti in cui questo fenomeno trova applicazioni interessanti sono molteplici: scienza della sicurezza (evacuazione di ambienti pubblici), progettazione architetturale, studio del flusso di traffico, studio sociologico dei comportamenti collettivi e anche intrattenimento. La dinamica delle folle è di grande interesse sia sotto condizioni critiche che normali. Nella fase di progettazione di ambienti chiusi come centri commerciali, stadi e scuole dove centinaia di persone si concentrano per scopi differenti bisogna garantire una via di uscita a tutti gli utenti nonostante il numero limitato di punti di uscita. In condizioni normali si può utilizzare per studiare il comportamento delle folle in ambienti come fiere, quartieri, o intere città, in modo, ad esempio, da agevolare la viabilità nei tratti più affollati.\\
Lo studio delle masse è un argomento molto affascinante, ma allo stesso tempo, molto complesso. Per simulare situazioni del mondo reale è spesso richiesta la modellazione di comportamenti collettivi, come anche individuali, di folle di grandi dimensioni che si muovono in ambienti anch'essi molto grandi. La complessità quindi si sviluppa in due direzioni: logica e temporale. Le simulazioni di modelli di grandi dimensioni, infatti, sono processi che richiedono molto tempo e grande potenza computazionale. Come è possibile ridurre il tempo di esecuzione di queste simulazioni? \\
Un possibile approccio (per risolvere questa criticità) è integrare la simulazione vera e propria con metodi analitici come le Catene di Markov. Si scompone il modello in porzioni di dimensioni minori e si eseguono simulazioni isolate su ognuno di questi, quindi si usano metodi matematici e analitici per mettere insieme questi risultati ed ottenere dei dati che cercano di avvicinarsi il più possibile a quelli che si sarebbero ottenuti eseguendo la simulazione sul modello completo.\\
Questo approccio, però, comporta una moltiplicazione delle simulazioni da eseguire su “patch” adiacenti che rischia di allungare ulteriormente il tempo necessario per una simulazione completa.\\
L'obiettivo perseguito in questa tesi, quindi, è ridurre i tempi di simulazione automatizzando la compilazione dei modelli NetLogo che eseguiranno le simulazioni.\\
In concreto il progetto consiste in un programma Java che riceve in ingresso un documento XML in cui è descritto il modello suddiviso in topologia, comportamenti possibili degli attori e stato iniziale dell'ambiente. In seguito alla costruzione degli oggetti Java necessari per la rappresentazione del modello viene eseguita la scrittura del codice NetLogo che modella l'ambiente inizialmente descritto. In questo modo si evita la scrittura da parte di un umano del codice NetLogo sicuramente meno semplice e intuitivo del XML.
 