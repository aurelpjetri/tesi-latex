\chapter{Introduzione}
NetLogo è un linguaggio di programmazione agent-based e un ambiente di modellazione di sistemi complessi. Permette la simulazione di fenomeni naturali e sociali, e si adatta molto bene alla simulazione di eventi complessi che si evolvono nel tempo. La natura del linguaggio di programmazione permette di specificare comandi a singoli agenti o a intere \textit{breeds}, definendo quindi \textit{behaviors} individuali e globali. In questo modo si possono studiare comportamenti a più livelli, come quello microscopico dei singoli individui, o quello macroscopico delle loro interazioni e dei pattern che ne possono emergere.\\
NetLogo come linguaggio di programmazione agent-based eredita da \textbf{Logo} il principio fondamentale " low threshold, no ceiling ". Low threshold implica la sua accessibilità a utenti poco esperti o con alcuna esperienza di programmazione precedente, questo perché Logo nasce come linguaggio di programmazione educativo. No ceiling, invece, si riferisce alla completa programmabilità di tutte le sue parti.\\
Come il suo predecessore, però, NetLogo è un compromesso tra linguaggio sequenziale e funzionale rendendolo, per alcuni aspetti, poco versatile ad occhi esperti. Lo scopo di questo progetto, quindi, è di agevolare l'utente NetLogo nella costruzione di particolari modelli. Si vuole facilitare la scrittura del codice NetLogo rappresenta ambienti fisici sotto forma di grafi, e che definisce tutti i possibili comportamenti che gli agenti possono avere al suo interno.\\
Il progetto prevede la lettura di un file XML in cui si descrive il modello di interesse, costituito da:
\begin{enumerate}
\item ambiente fisico
\item comportamento degli attori
\item stato iniziale 
\end{enumerate}
La natura funzionale di NetLogo fa sì che gran parte del codice per la rappresentazione dei modelli di interesse per questo progetto rimanga invariata, quindi l'obiettivo ultimo è quello di inserire le parti variabili che descrivono topologia, comportamenti e stato, all'interno di un codice fisso.
 