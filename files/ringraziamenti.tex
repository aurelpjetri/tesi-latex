\chapter{Ringraziamenti}

Ringrazio il professor Enrico Vicario per avermi dato la possibilità di svolgere questo lavoro di tesi su un argomento molto interessante. Un particolare ringraziamento va al mio co-relatore Sandro Mehic che con molta pazienza e costanza mi ha aiutato e insegnato tantissimo in questi mesi.

Ringrazio prima tra tutti mia sorella che mi ha sempre sostenuto e incoraggiato nei momenti più difficili diventando il mio punto di riferimento e facendo di me quello che sono oggi. Ringrazio mia madre e mio padre che con tantissimi sacrifici mi hanno permesso di raggiungere questo traguardo importantissimo. Un grazie di cuore va anche a Bernardo che non si è mai arreso nel darmi consigli e nello spingermi a provare cose nuove. Alla mia famiglia dedico tutto ciò che ho realizzato fino ad oggi, senza il loro sostegno non ce l'avrei mai fatta.