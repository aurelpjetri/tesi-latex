\chapter{Conclusioni e sviluppi futuri}
Abbiamo mostrato il funzionamento del componente Java sviluppato e illustrato il contesto in cui questo si è rivelato utile semplificando il processo di costruzione ed esecuzione delle simulazioni di folle. Gli esperimenti condotti ci hanno permesso di verificare che i suoi comportamenti rispettassero sempre le attese. Durante questi test sono stati misurati i tempi di simulazione necessari per l'esecuzione del modello monolitico e di due scenari in cui quest'ultimo è stato diviso in 6 e 12 regioni adiacenti. Da queste misurazioni si osserva che una maggiore suddivisione dell'ambiente riduce il tempo di simulazione totale. Un' altro vantaggio portato dalla scomposizione è la considerevole riduzione del tempo necessario per misurare gli effetti di modifiche apportate su punti localizzati della mappa. Infatti la scomposizione in regioni permette di ripetere le simulazioni solo sulla regione interessata, invece che su tutto l'ambiente. 

Nei modelli generati dal componente ci siamo focalizzati sull'aspetto fisico degli attori, modellando gli obiettivi che questi avevano all'interno dell'ambiente. Come lavoro futuro, quindi, sarebbe interessante estendere la logica rappresentativa dei comportamenti degli attori, includendo aspetti sociali come altruismo e conformismo, in modo da rendere i modelli generati ancora più vicini alla realtà. 

Un ulteriore sviluppo di questo progetto è sicuramente l'inclusione di nuovi parser in grado di analizzare formati diversi dall'XML, come ad esempio il CSV, e l'implementazione di una interfaccia, estendendo quindi i contesti in cui questo framework potrebbe rendersi utile e semplificandone l'utilizzo.