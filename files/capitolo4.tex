\chapter{Esperimenti}
Nella fase di sperimentazione del componente ci siamo posti come obiettivo quello di mettere a confronto la simulazione di un modello monolitico e completo con le simulazioni localizzate su specifiche sezioni in cui quest'ultimo è stato diviso. In particolare abbiamo deciso di confrontare il modello completo con due scenari diversi in cui quest'ultimo è stato diviso in 6 e 12 macro regioni adiacenti.\\
La mappa usata per gli esperimenti è tratta da un blocco residenziale della città di Firenze, in cui abbiamo cercato di inserire diverse topologie in modo da rappresentare zone con strade brevi e strette come il centro storico e strade di dimensioni maggiori come nelle moderne metropoli.\\
\section{Modello monolitico}
La struttura del modello monolitico è rappresentata in figura. %ref{figura modello monolitico} 

Lo strumento sviluppato in questo progetto di tesi è stato utilizzato dal dottor Sandro Mehic nelle attività di sperimentazione in \cite{esperimenti-sandro} per la raccolta dei tempi di soggiorno degli attori nelle singole regioni. Questi dati sono stati utilizzati per il calcolo delle probabilità di transizione delle Catene di Markov che astraggono il comportamento dell'attore attraverso le varie regioni in cui il modello è stato diviso.

dfghjkl
 